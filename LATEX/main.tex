\documentclass{article}

% Paquetes para mejor apariencia y personalización
\usepackage{amsmath}
\numberwithin{equation}{section}
\usepackage[utf8]{inputenc}  % Codificación de caracteres
\usepackage{graphicx}        % Incluir imágenes


\title{MÉTODOS DE DETERMINACIÓN DE ÓRBITAS A PARTIR DE OBSERVACIONES}   % Título del artículo
\author{
    Víctor Ávila Camargo, José Ignacio Miguel Rodríguez, Javier Zaragozano Calvo
}
\date{\today}  % Fecha del artículo, por defecto pone la fecha actual.

\begin{document}

% Generar la portada
\maketitle
%\includegraphics[width=1\textwidth]{portada.jpg}
\newpage

\begin{abstract}
    En este artículo se abordaran varios métodos para determinar 
    las órbitas de cuerpos celestes a partir de obervaciones 
proporcionadas por la NASA.
\end{abstract}
\newpage
\section{Introducción} %Motivacion del problema
\section{Método 1}
\section{Método 2}
\section{Método 3} %Creo que vamos a hacer 3 métodos 
\section{Resultados} %Aquí podríamos hablar de nuestros datos, el código y los propios resultados 
\section{Conclusiones} %A lo mejor podemos anañizar los resultados aquí

\end{document}