\documentclass{article}

% Paquetes para mejor apariencia y personalización
\usepackage{amsmath}
\numberwithin{equation}{section}
\usepackage[utf8]{inputenc}  % Codificación de caracteres
\usepackage{graphicx}        % Incluir imágenes
\usepackage{imakeidx}
\makeindex[options=-c, intoc]


\title{MÉTODOS DE DETERMINACIÓN DE ÓRBITAS A PARTIR DE OBSERVACIONES}   % Título del artículo
\author{
    Víctor Ávila Camargo, José Ignacio Miguel Rodríguez, Javier Zaragozano Calvo
}
\date{\today}  % Fecha del artículo, por defecto pone la fecha actual.

\begin{document}

% Generar la portada
\maketitle
%\includegraphics[width=1\textwidth]{portada.jpg}
\newpage

\tableofcontents

\newpage

\begin{abstract}
    En este artículo se abordarán varios métodos para determinar 
    las órbitas de cuerpos celestes a partir de obervaciones 
    proporcionadas por la NASA.
\end{abstract}
\newpage
\section{Introducción} %Motivación del problema
\index{Introducción}
Hoy en día y desde hace ya un tiempo, cada día se descubren 
varios exoplanetas cada uno con sus propias características. 
Para terminar de caracterizarlos por completo, es necesario 
hacer un estudio de cómo es su órbita. Esto es bastante 
importante porque estudiando su órbita podemos estimar algunos 
datos sobre ese exoplaneta que de otro modo sería imposible. 
Un ejemplo sería ver si se encuentra en la zona habitable 
o no de su estrella viendo qué tan grande es el periodo de su 
órbita. \\

Sin embargo, tomar una medida de su posición cada día sería 
algo tremendamente ineficiente. Es por eso que antiguos 
matemáticos y físicos desarrollaron algunos métodos para 
determinar su órbita de manera preliminar a partir de, generalmente, 
tres medidas. Algunos de estos métodos son el método de 
Gauss o el método de Gibbs. \\

Estos métodos han logrado varios hitos bastante importantes 
en la historia de la mecánica celeste. Uno de estos logros 
y quizás el más importante fue la determinación de la órbita 
del planeta enano Ceres. Este palneta enano fue descubierto 
el 1 de enero de 1801 por el astrónomo italiano Giuseppe Piazzi ($1746-1826$). 
Sin embargo, debdio a su rápido movimiento, pronto le perdieron 
de vista y ahí es cuando aparece el matemático Carl Friedrich Gauss ($1777-1855$) 
que, con su método, calculó la órbita de Ceres a partir de 
las pocas observaciones que había de él en la época y, gracias 
a su alta precisión, pudieron volver a localizarle. \\

Una vez motivado el esfuerzo de dedicarle un artículo a los 
método de determinación de órbitas, vamos a comenzar desarrollando 
el aspecto más teórico de los tres métodos en los que nos 
vamos a centar: El método de Gauss, el método de Gibbs y 
el método %El posible último método
\section{Método 1}
\index{Método 1}
\section{Método 2}
\index{Método 2}
\section{Método 3} %Creo que vamos a hacer 3 métodos 
\index{Método 3}
\section{Resultados} %Aquí podríamos hablar de nuestros datos, el código y los propios resultados
\index{Resultados} 
\section{Conclusiones} %A lo mejor podemos anañizar los resultados aquí
\index{Conclusiones}

\end{document}